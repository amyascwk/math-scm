\documentclass{article}

%Packages
\usepackage{amsmath,amssymb} %for math symbols

%Margins
\setlength{\oddsidemargin}{0cm}
\setlength{\topmargin}{0cm}
\setlength{\headheight}{0cm}
\setlength{\headsep}{0cm}
\addtolength{\textwidth}{4cm}
\addtolength{\textheight}{3cm}

%##############################################################################
%##############################################################################
\begin{document}
    
    \title{6.905 Final Project:\\Generic Abstract Mathematics}
    \author{Amyas Chew \and Lynn Chua \and Yongquan Lu}
    \maketitle
    
    %\tableofcontents
    
    %##########################################################################
    \section{Introduction}
        
        - Motivation
            - Role of computers in math
                - Computer Algebra Systems (CAS) take some computation load off user
                - Examples of proofs using computers: 4-color theorem, etc etc
            - Significance of project
                - Scheme allows emphasis on ideas rather than programming syntax
                - Want a flexible system allowing users to make own constructs
                - Free software and extensible so users can extend easily
                    - can keep in pace with real world progress in math community
        
        - Potential use cases
            - Convenience tool for keeping track of abstract mathematical data
            - Simple scripting language for a CAS backend
            - Possible basis for proof-assistants, proof-checkers, etc
            - ??
    
    
    %##########################################################################
    \section{Overview}
        
        - Overall structure
            - written using mit-scheme
            - Different levels of implementation:
                - Core
                - Math implementation
                - Extensions to base math
    
    
    %##########################################################################
    \section{Underlying structure}
        
        \subsection{Adapted infrastructure}
            
            - Generic operators
            - Other tools (symbolic arithmetic, amb, etc)
        
        
        \subsection{Math-object Datatype}
            
            - explanation of choice of structure
    
    
    %##########################################################################
    \section{Math Implementation}
        
        \subsection{Sets}
            
            - sets stuff
        
        
        \subsection{Group-like objects}
            
            - group-like stuff
        
        
        \subsection{Ring-like objects}
            
            - ring-like stuff
        
        
        \subsection{Hypergraph-like}
            
            - hypergraph-like stuff
    
    
    %##########################################################################
    \section{Specific constructors}
        
        \subsection{Group constructors}
            
            - group constructors and methods
    
    
    %##########################################################################
    \section{Code}
        
        The latest version of the code can be obtained from \texttt{https://github.com/amyascwk/math-scm}.
        
    
\end{document}